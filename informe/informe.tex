%%%%%%%%%%%%%%%%%%%%%%%%%%%%%%%%%%%%%%%%%
% University Assignment Title Page 
% LaTeX Template
% Version 1.0 (27/12/12)
%
% This template has been downloaded from:
% http://www.LaTeXTemplates.com
%
% Original author:
% WikiBooks (http://en.wikibooks.org/wiki/LaTeX/Title_Creation)
%
% License:
% CC BY-NC-SA 3.0 (http://creativecommons.org/licenses/by-nc-sa/3.0/)
% 
% Instructions for using this template:
% This title page is capable of being compiled as is. This is not useful for 
% including it in another document. To do this, you have two options: 
%
% 1) Copy/paste everything between \begin{document} and \end{document} 
% starting at \begin{titlepage} and paste this into another LaTeX file where you 
% want your title page.
% OR
% 2) Remove everything outside the \begin{titlepage} and \end{titlepage} and 
% move this file to the same directory as the LaTeX file you wish to add it to. 
% Then add \input{./title_page_1.tex} to your LaTeX file where you want your
% title page.
%
%%%%%%%%%%%%%%%%%%%%%%%%%%%%%%%%%%%%%%%%%

%----------------------------------------------------------------------------------------
%	PACKAGES AND OTHER DOCUMENT CONFIGURATIONS
%----------------------------------------------------------------------------------------

\documentclass[12pt]{article}
\usepackage[catalan]{babel}
\usepackage[utf8]{inputenc}
\usepackage{float}
\usepackage{graphicx}
\usepackage{wrapfig}
\usepackage{lscape}
\usepackage{rotating}
\usepackage{epstopdf}
\usepackage{makeidx}

\makeindex
\begin{document}

\begin{titlepage}

\newcommand{\HRule}{\rule{\linewidth}{0.5mm}} % Defines a new command for the horizontal lines, change thickness here

\center % Center everything on the page
 
%----------------------------------------------------------------------------------------
%	HEADING SECTIONS
%----------------------------------------------------------------------------------------

\textsc{\LARGE Universitat de Barcelona}\\[1.5cm] % Name of your university/college
\textsc{\Large Client-Servidor}\\[0.5cm] % Major heading such as course name
\textsc{\large Enfonsar la flota}\\[0.5cm] % Minor heading such as course title

%----------------------------------------------------------------------------------------
%	TITLE SECTION
%----------------------------------------------------------------------------------------

\HRule \\[0.4cm]
{ \huge \bfseries Pràctica 1}\\[0.4cm] % Title of your document
\HRule \\[1.5cm]
 
%----------------------------------------------------------------------------------------
%	AUTHOR SECTION
%----------------------------------------------------------------------------------------



% If you don't want a supervisor, uncomment the two lines below and remove the section above
\Large \emph{Author:}\\
Christian José \textsc{Soler}\\
Nicolás Martín \textsc{Forteza Ocaña}\\[3cm] % Your name

%----------------------------------------------------------------------------------------
%	DATE SECTION
%----------------------------------------------------------------------------------------

{\large \today}\\[3cm] % Date, change the \today to a set date if you want to be precise

%----------------------------------------------------------------------------------------
%	LOGO SECTION
%----------------------------------------------------------------------------------------

%\includegraphics{Logo}\\[1cm] % Include a department/university logo - this will require the graphicx package
 
%----------------------------------------------------------------------------------------

\vfill % Fill the rest of the page with whitespace

\end{titlepage}
\addcontentsline{toc}{section}{Índice alfabético}
\printindex
\section*{Introducció}
En aquesta pràctica se'ns demana implementar el joc d'Enfonsar la flota perquè juguin diferents Clients contra un mateix servidor.

L'objectiu del joc és endevinar la situació dels vaixells de l'enemic i enfonsar-los indicant les coordenades. Tot això entre un client i servidor que implementi el protocol especificat a l'enunciat.

L'objectiu d'implementar aquest programa és aprendre a utilitzar els mecanismes de programació Client/Servidor en Java. En concret aprendre a utilitzar l'API Socket de Java.net i a fer servidors multi-petició.

El Client haurà de tenir un mode manual i un mode automàtic per a jugar una única partida a l'hora, en canvi, el Server solament tindrà mode automàtic i podrà jugar diverses partides a l'hora gestionades en un cas per Threads i en un altre per un Selector.
Tot això s'haurà de fer fent un bon ús de la programació Orientada a Objectes en Java.

\newpage
\section*{Plantejament}
Inicialmente, tuvimos que plantear bien el proyecto y hacer un buen diagrama de clases, ya que eso nos permitió que nuestro código sea robusto  y a la larga, nos ahorró mucho tiempo y previno posibles errores.
Nuestro primer objetivo fue pues crear una versión local del juego, donde el cliente juega contra su propio tablero, pudiendo así probar la mecánica del juego básico, sin comunicación.

Una vez arreglados los errores, comenzamos a preparar la comunicación y simultáneamente empezamos a poner la base del servidor con Threads. En este punto nos dimos cuenta que estos proyectos compartían tanto código que merecía la pena crear una librería común, llamada swd-core,  donde vinieran todas aquellas clases que ambos usan. Eso se debe a que nos parecía poco óptimo tener que modificar el mismo código varias veces.

Como hemos ido aplicando bastantes patrones de diseño y creamos swd-core, hemos podido reducir  considerablemente el tiempo que necesitamos para hacer funcionar la comunicación entre cliente-servidor.

Hecho esto, pasamos a la última tarea de esta entrega: crear el servidor con selectores. Nos inspiramos en el ejemplo que se nos dio en clase.

Hicimos nuestras últimas pruebas y en la sesión de test algunas otras, con esto ya acabamos en trabajo.
\newpage
\section*{Annex}
\begin{sidewaysfigure}[ht]
    \includegraphics[width=\textwidth,height=\textheight,keepaspectratio]{../diagrams/class-diagrams/clientClasses.png}
    \caption{Client Diagram}
    \label{fig:PropProf}
\end{sidewaysfigure}
\begin{sidewaysfigure}[ht]
    \includegraphics[width=\textwidth,height=\textheight,keepaspectratio]{../diagrams/class-diagrams/threadsClasses.png}
    \caption{Threads-Server Diagram}
    \label{fig:PropProf}
\end{sidewaysfigure}
\begin{sidewaysfigure}[ht]
    \includegraphics[width=\textwidth,height=\textheight,keepaspectratio]{../diagrams/class-diagrams/selectorClasses.png}
    \caption{Selector-Server Diagram}
    \label{fig:PropProf}
\end{sidewaysfigure}

\end{document}	